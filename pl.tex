\documentclass[11pt,a4paper]{moderncv}
\moderncvstyle{classic}
\moderncvcolor{blue}
\usepackage[utf8]{inputenc}
\usepackage[scale=0.8]{geometry}

% Ustawienia stopki
\rfoot{Twoja stopka tutaj}

\renewcommand{\footrulewidth}{0.4pt} % Grubość linii oddzielającej stopkę od reszty strony

\name{Piotr}{Prochnicki}
\title{CV – Wsparcie IT}
\email{piotr.prochnicki@gmail.com}
\homepage{piotrit2015.github.io}
\social[linkedin]{piotr-prochnicki}

\begin{document}

\makecvtitle

\section*{Podsumowanie zawodowe}
Specjalizuję w systemach rodziny Linux, co potwierdzają ukończone kursy, ale zawodowo miałem do czynienia z migracjami głównie systemów 
operacyjnych Windows oraz rozwiązywaniu problemów z kompatybilnością. Podczas samodzielnego projektu 
migracyjnego z Windows 10 do 11 rozwiązałem problem z obsługą virtualizacji, także na sprzęcie domowym, co wymagało dogłębnej analizy BIOS-u. 
Moja wiedza techniczna (kurs CCNA) i podejście analityczne pozwalają mi skutecznie wspierać użytkowników w trudnych sytuacjach. 
Wnoszę wartość jako specjalista, który nie tylko zna teorię, ale potrafi ją zastosować w praktyce.

\section{Umiejętności}
\cvitem{Systemy}{Windows, Linux, znajomość podstaw sieci komputerowych}
\cvitem{Narzędzia}{Docker, LAMP, PHP (podstawy), pakiet MS Office, e-mail}
\cvitem{Wsparcie}{Zdalna pomoc techniczna, konfiguracja sprzętu i oprogramowania, obsługa użytkownika końcowego}
\cvitem{Miękkie}{Odpowiedzialność, terminowość, praca w zespole, dokumentowanie procesów}

\section{Doświadczenie zawodowe}
\cventry{05.2023 -- 09.2023}{Inżynier DevOps (staż)}{HXS Sp. z o.o. Sp.k.}{}{}{
  \begin{itemize}
    \item Instalacja LAMP i konfiguracja Docker.
    \item Opanowanie podstaw PHP.
  \end{itemize}
}

\cventry{12.2022 -- 03.2023}{Asystent Administracyjno-Biurowy (staż)}{Tax Unit Sp. z o.o.}{}{}{
  \begin{itemize}
    \item Archiwizacja dokumentów, obsługa pism urzędowych.
  \end{itemize}
}

\cventry{03.2022 -- 04.2022}{Pracownik sklepu internetowego (staż)}{Charity Polska}{}{}{
  \begin{itemize}
    \item Obsługa sklepu internetowego: pakowanie, wystawianie towaru.
  \end{itemize}
}

\cventry{09.2020 -- 12.2020}{Pracownik biurowy (staż)}{Grupa Center}{}{}{
  \begin{itemize}
    \item Obsługa oprogramowania sklepowego, opisy produktów.
  \end{itemize}
}

\cventry{11.2018 -- 06.2020}{Pracownik administracyjno-biurowy}{HR Team Sp. z o.o.}{}{}{
  \begin{itemize}
    \item Obsługa danych, e-maili, dokumentacji biurowej.
  \end{itemize}
}

\cventry{07.2017 -- 09.2017}{Cisco Services Engineer Intern}{Cisco Systems}{}{}{
  \begin{itemize}
    \item Praca w międzynarodowym zespole, Python, sieci komputerowe.
  \end{itemize}
}

\cventry{03.2017 -- 04.2017}{Informatyk – konserwator}{Stowarzyszenie COGITO}{}{}{
  \begin{itemize}
    \item Obsługa infrastruktury IT i lokalnej sieci.
  \end{itemize}
}

\cventry{10.2016 -- 12.2016}{Technik prac biurowych}{Urząd Miasta Krakowa}{}{}{
  \begin{itemize}
    \item Obsługa teleinformatyczna urzędu.
  \end{itemize}
}

\cventry{07.2015 -- 09.2015}{Technik prac biurowych}{Małopolski Urząd Wojewódzki}{}{}{
  \begin{itemize}
    \item Obsługa teleinformatyczna urzędu.
  \end{itemize}
}

\fancyfoot[LE,RO]{Wyrażam zgodę na przetwarzanie moich danych osobowych dla potrzeb niezbędnych do realizacji procesu rekrutacji zgodnie z Rozporządzeniem Parlamentu Europejskiego i Rady (UE) 2016/679 z dnia 27 kwietnia 2016 r. w sprawie ochrony osób fizycznych w związku z przetwarzaniem danych osobowych i w sprawie swobodnego przepływu takich danych oraz uchylenia dyrektywy 95/46/WE (RODO).}

\newpage

\section{Języki}
\cvitem{Polski}{ojczysty}
\cvitem{Angielski}{zaawansowany}
\cvitem{Francuski}{średnio-zaawansowany}

\section{Edukacja}
\cventry{2015 -- 2017}{Akademia Górniczo-Hutnicza, Kraków}{Informatyka Stosowana}{Systemy Informatyki Przemysłowej}{mgr inż.}{
  \begin{itemize}
    \item Tytuł pracy dyplomowej: "Analiza głównych składowych widm RXES" (Python).
  \end{itemize}
}

\cventry{2008 -- 2015}{Akademia Górniczo-Hutnicza, Kraków}{Fizyka Medyczna}{}{inż.}{
  \begin{itemize}
    \item Tytuł pracy dyplomowej: "Grafen i nanorurki węglowe: ich zastosowania oraz właściwości fizyko-chemiczne".
  \end{itemize}
}



\end{document}


